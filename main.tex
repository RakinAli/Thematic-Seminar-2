\documentclass[a4paper,11pt]{article}

\usepackage[english]{babel}
\usepackage[nottoc]{tocbibind}
\usepackage[a4paper, total={6.5in, 10in}]{geometry}

\usepackage{hyperref}
\hypersetup{
    colorlinks=true,
    linkcolor=blue,
    filecolor=magenta,      
    urlcolor=cyan,
    pdftitle={Overleaf Example},
    pdfpagemode=FullScreen,
    }

\urlstyle{same}

\usepackage[T1]{fontenc}
\usepackage[utf8]{inputenc}
\usepackage{todonotes}
\usepackage{listings} \lstset{language=c, showstringspaces=false, xleftmargin=-60pt, frame=l}

\usepackage{graphicx}
\usepackage{caption}
\usepackage{subcaption}
\usepackage{csquotes}


\usepackage{biblatex}
\addbibresource{refs.bib}



\title{Thematic Seminar 2 }
\author{Rakin Ali}

\begin{document}
\maketitle

\section{Introduction}
In this assignment, the \textit{\textbf{Food industry}} and \textit{\textbf{Buildings}} was chosen and is, briefly, discussed in this report. The report will answer the following questions:

\begin{itemize}
  \item What/Which is the problem
  \item Where in the system does it emerge and why? 
  \item What is your proposed ICT solution and in what way could it contribute to decrease impact?
  \item Are there any possible negative indirect consequences of your solution?
\end{itemize}

\section{What/Which is the problem?} 
\subsection{Food}
Roughly one third of the food produced in the world for human consumption goes to waste or gets lost. \cite{Website:UNEP} Decreasing food waste is one of \textit{United Nations Sustainable Development Goals(section 12). 
} \cite{website:UnitedNations}.
\subsection{Buildings}
Avoidable expenditure on energy is used to heat buildings at times when it's not being occupied. Thermostat is responsible for half of each households energy bill which can be avoided\cite{website:Nest}. Reducing enviromental affects from settlements relates to United Nation Sustainable development goals \textit{(section 11)}.


\section{Where in the system does it emerge and why?}
\subsection{Food}
Food waste is defined as either \textit{food lost} or \textit{food waste}. Food loss occurs more often and is any edible food that goes uneaten at any stage. Food waste is a specific kind of food loss. It is food discarded by retailers due to color or appearance and plate waste by consumers according to USDA\cite{website:USA}. Edible food is discarded at every point of along the supply chain\cite{NYtimes}. Food lost occurs on fishing boats, transportation and distribution to the retailers and supermarkets, waste in retail businesses, restaurants and households. To summarise either food lost or waste happens in all stages of the food system \cite{website:FoodPrint}.

\subsection{Buildings} In short summary, most
thermostats in buildings are not able to optimize and predict energy usages \cite{website:Energy-saver}. 

\section{What is your proposed ICT solution and in what way could it contribute to decrease impact?}
\subsection{Food}
Having previously worked in many restaurants and Supermarkets, immeasurable food waste has been personally observed. Perfectly edible food that contains no, or minimal, health risks and are hygienically clean that could've been sold but not done because of "selling hours" was finished, overproduced or didn't appear in an ascetic way that matched the restaurant/Supermarkets view of the food. An app named \textbf{\textit{Karma}} \cite{website:Karma} tries to minimize this type of food lost by letting retailers resell the food not sold during retail hours to customers for a hefty discounted price. Customers are incentived to consume these products and all the customers have to do is appear to the re-seller and pick up the food. Retailers profit out of this while contributing to sustainable development and are able to empty the shelves which in turn adds more space for raw materials in the shelves and that in turn can generate even more profits. The first order impact is the amount of energy the mobile-app uses however this could be neglected as minimal changes that reduce the energy-cost of the app would not be noticed as the app in not running on the background all day. The second order is that the app reduces food lost as it provides incentives to eat perfectly edible left-over food. 
\subsection{Building}
Applying Machine Learning to thermostats so that they learn from your specific habits and usage to then be able to predict it would be useful. 

\section{Are there any possible negative indirect consequences of your solution?}
\subsection{Food}
The mobile app does not oversee the condition of the food the resells promotes on the app. Food that otherwise wouldn't pass food safety rules could theoretically be sold on the app which could affect the health of the consumer. Newly opened restaurants with unacceptable practices from an environmental perspective could promote their unsold food on the app which in turn raises awareness of the restaurants and that in turn generates more profit for these restaurants. Overall this app has a great idea however it should regulate what is sold and what food is promoted so that foods that pose a health risk are not sold.
\subsection{Buildings}
Data gathering in general raises ethical questions of privacy. The data could be used by hackers to predict when the consumer is at home and sell this data to the highest bidder on the internet. Building these thermostats would require advanced hardware which could make these thermostat unjustifiably expensive for regular homeowners. 

\printbibliography


\end{document}